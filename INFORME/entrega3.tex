\part{Entrega 3}

\section{Introducción}

Este proyecto busca diseñar un datacenter orbital de 1MW para entrenar modelos de inteligencia artificial usando paneles solares en órbita. Esto se basara en la idea presentada por Lumen Orbit que busca realizar esto para datacenters de 40MW.

El objetivo de este informe es presentar dos diseños de estrcuturas para este datacenter, de forma de cumplir ciertas especificaciones de diseño y de operación. Donde los principales requerimientos son la capacidad de generar 1MW de energía, para lo cual se necesitaran de 3.334 $m^2$ de paneles solares. Además, se deben cumplir ciertas condiciones de diseño, como que la fracción de masa debe ser menor al 20\%, la frecuencia del primer modo debe ser mayor a 0.1Hz, la desangulación para un cambio de temperatura de 150\textdegree{}C debe ser menor a 2\textdegree{} y se debe cumplir con un factor de seguridad de 2.

Dentro de las especificaiones del satelite, este sera un rectangulo de 6.6 m x 2.6 m x 7.8 m y el material de la estructura de los paneles solares sera de fibra de carbono de alto modulo M55J. Dentro de las consideraciones el peso de al estrcutura y satelite no se consideranran como un priblema, solo que la estructura sea lo suficientemente rigida para soportar una aceleracion de 0.1g en cualquier direccion. 

Para llevar a cabo este estudio se utilizara el software de diseño de estructuras OpenSeesPy, el cual permite realizar análisis de elementos finitos de estructuras. Además, se utilizaran otros para la visualización de los resultados obtenidos. El objetivo de cada diseño es cumplir con los requerimientos de diseño y operación, y a la vez minimizar el peso total y el momento de inercia de todo este datacenter.