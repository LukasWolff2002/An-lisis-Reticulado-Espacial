\part{Entrega 2}

\newpage
\section{Lukas tu codigo vale callampa, haz alguna wea, parte el informe, con mucho cariño ChatGPT}

\newpage
\section{Introducción}

Este informe presenta un análisis estructural de un enrejado simplificado respecto a la entrega anterior, considerando ahora la sección transversal más pequeña (\(A_{10}\)). Se evaluarán dos configuraciones de enrejado: una sin diagonales y otra con una sola diagonal en las caras laterales e interiores. El objetivo es analizar el comportamiento de las deformaciones y las fuerzas axiales en condiciones de carga menos exigentes: sin aceleración y con una disminución de temperatura de 50°C en los nodos.

El análisis está dividido en dos partes. La primera evalúa las deformaciones y fuerzas axiales con una reducción de temperatura, detallando la distribución de las fuerzas axiales y los desplazamientos de la estructura. La segunda parte estudia la "mejor dirección" de aceleración, definida como la dirección en la que se minimiza la carga axial en las barras.

\section{Resultados}

\subsection{Parte 1}

Se utilizó un reticulado de 8 vanos para una visualización más clara de los resultados. Los análisis de deformación y esfuerzos internos se realizaron para una temperatura de -50°C en las direcciones $x$, $y$, y $z$. Los resultados se muestran a continuación.

\subsubsection{Temperatura en Eje X}

\section*{Temperatura en Eje X (lukas ponlos tu no los encontre)}

\subsubsection{Temperatura en Eje Y}

\section*{Temperatura en Eje Y (lukas ponlos tu no los encontre)}

\subsubsection{Temperatura en Eje Z}

\section*{Temperatura en Eje Z (lukas ponlos tu no los encontre)}

\subsection{Parte 2}

Claro:
Para esta parte, se consideró una aceleración muy baja, con el objetivo de observar cómo las barras responden en una "mejor dirección", en la cual las cargas axiales son mínimas.

En esta parte cambie todo lo que me comentaste, si quieres hacer algún cambio adicional, házmelo saber.

\section*{Momento maximo (lukas ponlos tu no los encontre)}

Las tensiones más bajas se observan en las barras externas, que apenas se ven afectadas. Las barras interiores muestran tensiones ligeramente más altas, indicando que podrían no requerir refuerzos adicionales.

\section{Conclusión}

Hola aquí va tu conclusión:

El análisis de la configuración con una sola diagonal sugiere que las tensiones y deformaciones son mínimas en condiciones de carga bajas, como se esperaba. La estructura sin diagonales, aunque menos efectiva, presenta un comportamiento adecuado para estas condiciones reducidas. Finalmente, en la "mejor dirección" de aceleración, los esfuerzos axiales son mínimos en toda la estructura, especialmente en las barras exterio

Cualquier cambio que quieras hacer, házmelo saber.