\part{Entrega 1}
Claro, aquí tienes el texto corregido:

\section{Introducción}
Las estructuras que se utilizan en el espacio deben contar con una serie de características que les permitan operar y cumplir con su función de forma eficiente y segura. Una de estas características es la relación de masa estructural y la longitud de la estructura con las frecuencias y modos de vibración de la misma. Si estos requisitos no se cumplen, mandar la estructura al espacio además de ser costoso, puede ser peligroso para la tripulación y la misión.

En este informe se estudiarán varias relaciones de masa estructural para obtener un área óptima de un reticulado, junto con la determinación de los modos de vibración que sean mayores a 0.1 Hz, ya que esta es una frecuencia que se considera segura en el espacio.

Para cumplir con este objetivo, se emplearán la librería OpenSeesPy para el análisis estructural y la identificación de las frecuencias y modos de vibración de la estructura, y PyVista para la visualización de los datos obtenidos.
\newpage
\section{Modelo}


\section{Resultados}

\section{Conclusiones}